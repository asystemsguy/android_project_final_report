%%%%%%%%%%%%%%%%%%%%%%%%%%%%%%%%%%%%%%%%%%%%%%%%
\section{Problem}
\label{motivation}
%%%%%%%%%%%%%%%%%%%%%%%%%%%%%%%%%%%%%%%%%%%%%%%%


%Why we are building this system - motivation
One of the main requirements for an Android application is energy efficiency. Current energy management techniques in android give a little scope for energy control of applications based on user preferences. In this project, we develop a framework based on Android to enable user-centric energy management. This framework will enable users to give priority to the energy consumption of apps that are of high value to them. Furthermore, it will also give incentive for developers to create applications that are energy efficient. 


%Who are stake holders and what they want
There are three stakeholders that are mainly affected by this framework

\begin{enumerate}

\item Android end users
\item Android App developers
\item Smartphone device manufactures

\end{enumerate}

%What we are building
Users have a limited power in their Smartphone batteries. Among many applications installed by a user on their Smartphone, only a few frequently used applications are more important for them. For a user who is mindful of his battery life may want to restrict applications that have a high energy usage but has a little value for the user. To restrict such apps from consuming all of the energy, a user can use our framework to provide a priority for each app installed on their Smartphone. A user can give either high, medium, or low priorities for an app based on their preference. By default, applications will be in low priority when they are initially installed in the phone. 

High priority applications and the Android operating system itself are considered to be most important applications for the user and are allowed to unrestricted use of the energy. Energy usage of medium and low priority applications are considered to be less important for the user and energy usage of these applications are restricted by our framework. 

Android application developers should adapt their lower priority applications to make them more energy efficient. Energy used by applications will be accounted for and certain system resources will be denied when they use more energy than allowed by its priority. Application developers should accommodate these new insufficient energy errors and build their app accordingly to work for low, medium and high priorities. Some of the current applications such as Facebook have both full-fledged and lite applications for different kinds of phones and networks.

Accurate energy accounting depends on specific device and battery used. Hence the framework should be customizable for device manufacturers so that they can optimize for their devices and battery type used. 


%What are constriants of our system
Keeping various stakeholders in mind, our system should always ensure following properties

\begin{enumerate}
\item Always higher priority applications can use more energy than a lower priority application.

Our framework's main functionality is to restrict energy used by lower priority applications. On the other hand, it should allow unrestricted access to energy for high priority and system apps.

\item Fairness in-terms of energy availability among all applications belonging to same priority should be ensured.

If two applications are having same priority they both should have access to the same amount of energy. This may not guarantee two applications actually use the same amount of energy. 

For example, if App 1 comes into the system when the battery is completely charged and executes until the end;  App 2 has same priority but comes when the battery is completely drained it may not use the same amount of energy due to unavailability of energy in the battery. But in both cases framework should not restrict the app from using the same amount of energy if it is available.

\item The framework should consider the changes in total available energy as the battery drains over the time.

As total available energy changes framework should further tighten restrictions on lower priority applications. This will ensure high priority applications and android system can run longer. 

\item It should also take into account the changes in the number of energy consumers that are active in the Smartphone. 

Android applications will dynamically enter the system and leave the system. Framework should ensure new consumers are not effecting existing ones and fairness is ensured among all applications that are running.

\item The framework should have minimal overhead.
\item The framework should ensure failures due to lack of energy are graceful and developers are notified.

This is required for developers to understand why their application failed and write code to wait for energy availability before executing next operation.

\item The framework shouldn't reclaim energy once allocated to an app.  

The developer decides which system resources to use and which to avoid based on energy availability.  It will be hard for developers to decide which system resources to use if allowed energy limit set by the framework is not guaranteed.

\item Apps should benefit when system gains energy due to recharge of the battery.
\item The system should be customizable for different hardware devices and battery.

\end{enumerate}


In the next section, we will look at how system is designed to satisfy these properties.


\section{Solution}



\subsection{System design}


In this section, we will introduce unit of measurement for energy in our framework, how we account for energy consumption by the applications and finally, we will describe how we allocate energy between applications to ensure system properties are met.

\subsection{Unit of measurement}

An abstract unit of energy in our system is called energy credit. Having an abstracted unit will help us to make the framework more customizable for Android device makers. By default, an energy credit is equivalent to 1 mAh in the current prototype implementation. When a resource of an Android system such as the network or the disk is used by an application, it will spend some energy credits equivalent to the amount of energy spent in executing that system functionality.

\subsection{Energy accounting for system resources }

Android uses Linux kernel to manage system resources. In Linux kernel, system calls are the single point of contact for all user-level applications. Any user functionality that requires system resources such as network can only be executed by calling corresponding system calls. Using this fact, there are many previous android energy modeling techniques such as Eprof, which can calculate the energy consumption for each system call. In our framework, we will estimate amount of energy required to execute a system call and allow system call to execute only when required energy for application is available.

System calls can have either constant or varying energy cost. System calls such as socket::connect has a constant energy cost and socket::send has an energy cost depending on the amount of data it needs to send. Constant cost system calls can be estimated before hand and can be stored in a hash table. Verifying system calls with dynamic cost require a model to predict cost based some variables such as amount of data to be transfered as in the case of socket::send. These models can be different for different devices, we can create these models using values estimated using Eprof based on different parameters.

\subsection{ Allocation of credits }

Energy credit allocation among applications in a dynamic system such as Android can be tricky. The total number of available energy credits change as the battery is discharged continuously. Number applications consuming energy will also change with time. Furthermore, some applications will only be installed in the system, but may never be used. Hence it's important to carefully allocate energy credits to ensure all of the system properties stated in the section~\ref{motivation}  are met.

The lithium-ion batteries used in modern Smartphones have the capacity range from 1000 mAh to 5000mAh. For example, If a phone is fully charged and the battery has a capacity of 2000mAh and if applications has a continuous load of 200 mA, then the battery will last for 10 hours. By decreasing load on the battery, we can decrease overall power consumption and increase battery life. In the previous example, if applications only has a load of 100 mA then the battery will last for 20 hours.  To facilitate this we need to limit the amount of energy spent by applications. In our framework, we will limit number energy credits an low priority application can spend in a time period. In the prototype implementation, we had the time period as 1 min.

Our system should limit energy consumption of low and medium android applications without affecting the energy consumption of the system and high priority applications. To ensure this, we will allocate an infinite number of credits for high priority and system apps. For medium and low priority applications, we will allocate 0.1\% and 0.05\% of total available credits respectively per time period. 

\subsection{Example}
Below is the example on how it will work
%Example 

Our example smart phone battery have 2000 mAh as capacity for fully charged battery. if time period is 20 secs and limit for energy is 3000mAs for low priority app. If operations as specified in example given by Pathak et al. are executed in the below order

\begin{enumerate}
\item Cost of disk operation is 400mA and ran for 3 secs
\item Cost of small network transfer is 125mA and ran for 7 sec.
\item Cost of large network transfer is 500mA and ran for 10 sec.
\end{enumerate}

Execution of system calls:

\begin{itemize}

\item Then number of credits remaining after 1st operation ran will be 3000 - (400mA*3) = 1800mA

\item Number of credits that will be spent on small network transfer will be 1800 - (125*7) = 925

At this point of time, application don't have enough credits to continue and it can execute next system call (costs 5000 mAs) for network transfer only when app can be at higher priority.

\item System call for large network transfer will fail and app developer will be force to wait until next time period (to get enough credits back) to execute series of small data transfers which are less costly than a large network transfer.

\end{itemize}

\subsection{How we met system properties}

\begin{enumerate}

%How will ensure system properties are meet

\item  This system will ensure that lower priority applications always has limited energy usage. This will force applications developers to create applications to have a minimal energy consumption. 

\item  It will always allocate applications with same priority same amount of energy credits.

\item  Rate is proportional to amount of battery available 

\end{enumerate}

\subsection{Limitations}

\begin{enumerate}

\item  The number of credits required for a system call is approximation of original value.
\item  Battery characteristics such as charge will change over the lifetime of the battery. 
\item  Certain existing operations which require high energy may fail.
\item  In cases such as tail energy applications can have energy impact without calling system calls.

\end{enumerate}

\subsection{Assumptions}

Following are assumptions we make in the Android system
\begin{enumerate}

\item  We account for all energy consumption with system calls.
\item  Apps either run in background or foreground are considered active and consume energy.
\item  Inactive apps don't consume energy.
\item  The total amount of available energy is constant for the time period.

\end{enumerate}




















