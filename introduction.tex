%%%%%%%%%%%%%%%%%%%%%%%%%%%%%%%%%%%%%%%%%%%%%%%%
\section{Introduction}
\label{intro}
%%%%%%%%%%%%%%%%%%%%%%%%%%%%%%%%%%%%%%%%%%%%%%%%


Battery-driven devices such as smart phones, tablets and wearables are now commonplace in our lives. Android has become one of the dominant mobile platforms used in these devices. Android app repositories, such as Google Play, have created a fundamental shift in the way software is delivered to consumers, with thousands of apps added and updated on a daily basis. Various mobile app markets offer a wide range of apps from entertainment, business, health care and social life. Mobile phones today are filled with large number of apps. On an average each person contains 35 apps in a smart phone~\cite{thinkwithgoogle}. 

The capability of the smartphones has been increasing due to the improving computation power and inreasing number of sensors embedded in them. The modern applications targeted for the smartphones make use of more computation and sensors to give users more features for improved user experiences  and hence consume more power. From a user’s perspective, this produces tangible and pertinent problems. The use of energy-draining apps could quickly leave a user with empty battery, preventing from using the smart-phone even for phone calls. In addition, having and running such apps might require frequent battery re-charges. This represents a problem because modern battery’s life is quite limited, often to a finite amount of charging cycles (for Lithium-ion batteries), ranging between 300 and 500 cycles (with only 100-200 cycles after a mid-life point) and gradually decreasing with time~\cite{linares2014mining}. The usefulness of the Smartphone is now limited by capacity of the battery.

%edit  and citation required
The energy density of lithium-ion batteries used in Smartphones has only grown by four times since 1991 which is not in proportion with increasing rate of battery consumptions by the apps on the mobile and this not expected to grow any faster in near future. So, it becomes important to optimize usage of the battery and give control to user to decide so that user have preference of spending the mobile battery. Android OS introduced new features  like Doze, Standby modes, background service execution limit etc. to optimize the battery life. However, Android OS does not provide any control to user to choose the preferences.

Though many apps are installed in a phone, research on usage patterns of mobile applications in general media~\cite{techcrunch85}~\cite{techcrunch63} suggests that users use only few apps frequently. These important set of applications have high priority from user's perspective.But remaining apps installed in the phone also consume energy for background process such as ads~\cite{stevens2012investigating}. But in today's Android system, there is no feature to give priorities for certain apps and restrict power consumption by other less important apps. 

% This needs revision----
In this work, we present EnergyMAC: Energy based Access Control which is user centric energy management for controlling energy usage by the apps as per the user preference. In this, the user gets to choose priority of apps: High, Medium and Low and based on this priority, energy usage will be restricted.  This will also motivate app developer to think in terms of the energy savings, design and develop energy efficient app


There has been tons of work on optimizing the energy efficiency of the hardware components batching the operations like network. Users who wish to prolong their iPhone’s battery life can enable Low Power Mode.%cite 

%cite
There has some body of works on the offloading the tasks from the mobile device to the cloud to execute computation heavy tasks so that user can use mobile for longer time

% add previous works
