%%%%%%%%%%%%%%%%%%%%%%%%%%%%%%%%%%%%%%%%%%%%%%%%
\section{Introduction}
\label{intro}
%%%%%%%%%%%%%%%%%%%%%%%%%%%%%%%%%%%%%%%%%%%%%%%%


Battery-driven devices such as smart phones, tablets and wearables are now commonplace in our lives. Android has become one of the dominant mobile platforms used in these devices. Android app repositories, such as Google Play, have created a fundamental shift in the way software is delivered to consumers, with thousands of apps added and updated on a daily basis. Various mobile app markets offer a wide range of apps from entertainment, business, health care and social life. Mobile phones today are filled with large number of apps. On an average each person contains 35 apps in a smart phone~\cite{thinkwithgoogle}. 

The capability of the smartphones has been increasing due to the improvement in  computation power and increasing number of sensors embedded in them. In the meantime, modern mobile applications targeted for the smartphones make use of more computation and sensors to give users more features for improved user experiences  and hence consume more power. From a user’s perspective, this produces tangible and pertinent problems. The use of energy-draining apps could quickly leave a user with empty battery, preventing from using the smart-phone even for phone calls. In addition, having and running such apps might require frequent battery re-charges. This represents a problem because modern battery’s life is quite limited, often to a finite amount of charging cycles (for Lithium-ion batteries), ranging between 300 and 500 cycles (with only 100-200 cycles after a mid-life point) and gradually decreasing with time~\cite{linares2014mining}. The expected lifetime of the system is an important factor in the total user experience ~\cite{kim2013event} ~\cite{li2013energy}

%edit  and citation required
The usefulness of smartphone is now limited by capacity of the battery.
Energy density of lithium-ion batteries used in smartphones has only grown by four times since 1991~\cite{janek2016solid} which is not in proportion with increasing rate of battery consumptions by the hardwares and softwares component on the mobile and this not expected to grow any faster in near future. To close this gap, there has been several efforts from the researchers in academia and industry at different  layers of abstraction i.e. hardware, operating system,  applications  and human interactions to use the energy efficiently. Offloading computation intensive tasks from mobile devices to cloud to save energy consumption on the battery have been found to be successful~\cite{lai2017furion} ~\cite{boos2016flashback}. Though all these optimizations are done in the interests of  users however user preference has not been much taken into consideration.
 
 
 Android OS introduced new features Doze, Standby modes, background service execution limit etc. to optimize the battery life.In Doze mode, the system attempts to conserve battery by restricting apps' access to network and CPU-intensive services. It also prevents apps from accessing the network and defers their jobs, syncs, and standard alarms. In App Standby mode, network access is restricted and pending syncs and jobs are deffered for unused applications. In the recent Android OS update to 8.0, the system places restrictions on the app running in the background. A lot of efforts has been put in attributing expended system energy to apps~\cite{pathak2012energy}~\cite{yoon2012appscope}~\cite{zhang2010accurate}. This has helped the app developers in the spotting energy bugs, hotspots\cite{banerjee2014detecting} and insights like batching network operations~\cite{pathak2012energy} to optimize energy usage and motivated them to develop energy efficient apps. OS  treats all the app fairly but user might have preference of the apps as per need at different time.
 

One important aspect that can prove to be critical to user experience and extending the battery life as per the user's need is to give control to user to prioritize the mobile apps.Though many apps are installed in a phone, research on usage patterns of mobile applications in general media~\cite{techcrunch85}~\cite{techcrunch63} suggests that users use only few apps frequently. These important set of applications have high priority from user's perspective.But remaining apps installed in the phone also consume energy for background process such as ads~\cite{stevens2012investigating}. But in today's Android system, there is no feature to give priorities for certain apps and restrict power consumption by other less important apps. 

In this work, we present EnergyMAC: Energy based Access Control which is user centric energy management for controlling energy usage by the apps as per the user preference. In this, the user gets to choose priority of apps: High, Medium and Low and based on this priority, energy usage will be restricted.  This will also motivate app developer to think in terms of the energy savings, design and develop energy efficient app.

