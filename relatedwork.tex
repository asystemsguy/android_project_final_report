%%%%%%%%%%%%%%%%%%%%%%%%%%%%%%%%%%%%%%%%
%\vspace{0.1in}
\section{Related work}
\label{related}
%%%%%%%%%%%%%%%%%%%%%%%%%%%%%%%%%%%%%%%%

Energy accounting and management has been studies in the past.
There is a large body of work on energy consumption of Android apps. Prior related studies can be categorized in two ways: power modeling and power measurement. Research in power modeling suggests estimating the energy usage of mobile devices or apps in the absence of hardware power
monitors ~\cite{hao2013estimating}, ~\cite{li2013calculating}. These software-based approaches build models and capture model parameters from programs using static-analysis techniques. EnergyMac on other hand uses these models to enforce strict energy access limits on applications.

Studies in power measurement make use of specialized hardware, such as Monsoon~\cite{yoon2012appscope}, and map the sampled measurements to execution traces to determine an app's energy consumption at various granularities. To the best of our knowledge, EnergyMac is the first work that has attempted to implement a access control for Android apps according to their energy consumption. 

Both EcoSystem and Cinder tried to control the power consumption of applications accurately through managing battery energy as one kind of system resources.

%% summarize previous works and how our ideas differ the previous works








