\section{Evaluation}\label{evaluation}
In this section, we evaluate the performance overhead due to the control mechanism we have implemented in android kernel. For experiment, we used Goldfish Kernel on a emulator for testing. 

\subsection{Methodology}
In our experimental setup, we download files of different sizes i.e. 5 MB, 10MB, 20MB, 50MB from server[https://www.thinkbroadband.com/download ] using our custom android app on the Android emulator. This experiment is conducted for two cases (i) Emulator running original/unmodified Goldfish kernel  (ii) Emulator running Goldfish kernel with our modifications for control mechanism. 

%% figure here 
\subsection{Results}
We conduct the evaluation by running the experiment 20 times for each file size as mentioned in the previous section and record the time taken to download and plot the data collected from the experiment. Figure {} presents our experimental results. From the figure, it can be noted that mean overhead due to our modification is minimum i.e. under 2 sec and overhead would not be visible to the user. The varying download time difference for each file size in each  could be due to the fluctuation in the network latency.
In future, we plan to experiment using other operations which are least affected by network fluctuations or any other factors.



