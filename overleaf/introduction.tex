%%%%%%%%%%%%%%%%%%%%%%%%%%%%%%%%%%%%%%%%%%%%%%%%
\section{Introduction}
\label{intro}
%%%%%%%%%%%%%%%%%%%%%%%%%%%%%%%%%%%%%%%%%%%%%%%%

Microservices ~\cite{DBLP:journals/corr/DragoniGLMMMS16} a new trend in architecting large software systems wherein a system is designed as a set of microservices. Microservices can be developed, managed and scaled independently. There is typically some kind of routing fabric ~\cite{Selimi:2017:PSP:3101112.3101167} that gets requests to a specific instance of a microservices; this routing fabric often provides load-balancing and can isolate microservices that are in a failed state. There are costs to this approach as well such as the computational overhead of running an application in different processes and having to pay network communication costs ~\cite{7742218} rather than simply making function calls within a process.

Using Web standards is recognized as a common approach in building microservices application architectures. REST mechanism is widely used for data-interchange ~\cite{DBLP:journals/corr/DragoniGLMMMS16} in microservices. It is a useful integration method because of its comparatively lower complexity over other protocols. Some of the basic properties of a REST protocol is statelessness and cache-ability. These properties will facilitate performance and reliability improvements in applications using REST.

For an efficient communication between microservices using REST, we need certain basic services like name-resolution of service endpoints, load balancing between different instances of the same service and caching of REST objects. Currently, these services are not provided by the network, But they are provided by external systems like DNS and middleboxes like HTTP cache. This middleware can introduce extra latency to API call under certain conditions such as cache miss event.  

The Named Data Networking (NDN) ~\cite{zhang2014named} project aims to develop a new Internet architecture that can capitalize on strengths and address weaknesses of the Internet’s current host-based, point-to-point communication architecture in order to naturally accommodate emerging patterns of communication. By naming data instead of their locations, NDN transforms data into a first-class entity. It also enables several radically scalable communication mechanisms such as automatic caching to optimize bandwidth.

In this project, we will apply NDN features like named routing and automatic caching using in-network storage to create a REST implementation on NDN without middleboxes that can introduce latencies. We will evaluate our implementation by showing the reduced end-to-end latency of a REST API in our sample microservices application.

In section ~\ref{related}, we will introduce to existing solutions and NDN architecture. In section ~\ref{motivation}, we will give problem statement and motivation for the problem. Section ~\ref{solution} will contain a description of the proposed solution. Finally, in ~\ref{evaluation}, we will specify our evaluation strategy. 



